\def\tightlist{}
\documentclass[biblatex,ddraft]{deltares_report}
% \usepackage{booktabs}
% \usepackage{tabu}
$if(highlighting-macros)$
$highlighting-macros$
$endif$
$if(csl-refs)$
% Pandoc citation processing
\newlength{\cslhangindent}
\setlength{\cslhangindent}{1.5em}
\newlength{\csllabelwidth}
\setlength{\csllabelwidth}{3em}
\newlength{\cslentryspacingunit} % times entry-spacing
\setlength{\cslentryspacingunit}{\parskip}
% for Pandoc 2.8 to 2.10.1
\newenvironment{cslreferences}%
  {$if(csl-hanging-indent)$\setlength{\parindent}{0pt}%
  \everypar{\setlength{\hangindent}{\cslhangindent}}\ignorespaces$endif$}%
  {\par}
% For Pandoc 2.11+
\newenvironment{CSLReferences}[2] % #1 hanging-ident, #2 entry spacing
 {% don't indent paragraphs
  \setlength{\parindent}{0pt}
  % turn on hanging indent if param 1 is 1
  \ifodd #1
  \let\oldpar\par
  \def\par{\hangindent=\cslhangindent\oldpar}
  \fi
  % set entry spacing
  \setlength{\parskip}{#2\cslentryspacingunit}
 }%
 {}
\usepackage{calc}
\newcommand{\CSLBlock}[1]{#1\hfill\break}
\newcommand{\CSLLeftMargin}[1]{\parbox[t]{\csllabelwidth}{#1}}
\newcommand{\CSLRightInline}[1]{\parbox[t]{\linewidth - \csllabelwidth}{#1}\break}
\newcommand{\CSLIndent}[1]{\hspace{\cslhangindent}#1}
$endif$
\begin{document}
\title{Zeespiegelmonitor}
\subtitle{2022}
\author{Deltares}
% \partner{Partners}

\client{Rijkswaterstaat WVL}
\contact{Saskia van Gool}
% \reference{Reference}
\keywords{Zeespiegel, Zeespiegelstijging, Bodemdaling}

\version{0.5}
\date{15-09-2022}
% \projectnumber{11208038}
% \documentid{Id}
\status{concept}
% \disclaimer{Disclaimer}
%\references{References}

\authori{Willem Stolte}
\organisationi{Deltares}
\authorii{Fedor Baart}
\organisationii{Deltares}
\authoriii{Sanne Muis}
\organisationiii{Deltares}
\versioni{0.5}
\revieweri{Bart van den Hurk}
\approvali{Paul Saager}
\publisheri{Deltares}

\summary{
Sinds 2014 onderhoudt Deltares in opdracht van Ministerie van I\&W de Zeespiegelmonitor. Doel is om, op basis van de waterstandsdata van de zes Nederlandse hoofdkuststations, de stand en ontwikkeling van de zeespiegel vast te stellen. Deze worden gebruikt om de uitvoering van het waterveiligheidsbeleid te ondersteunen. Het gaat in het bijzonder om de gemiddelde hoeveelheid jaarlijks te suppleren zand en toetsingen en ontwerpen van keringen. De gegevens van de Zeespiegelmonitor worden ook gebruikt in het opvolgen van vergunningen van gas- en zoutwinning onder de Waddenzee.

De stand van de zeespiegel wordt elk jaar vastgesteld. Ze is het resultaat van daadwerkelijke (absolute) stijging van de zeespiegel en bodemdaling. Hierom wordt gesproken over relatieve zeespiegelstijging. Iedere drie jaar is er een rapportage die ingaat op de waarnemingen, de conclusies daaruit en de resultaten van onderzoek ten behoeve van de Zeespiegelmonitor. Dit is de derde rapportage.

Voor het bepalen van de stand en trend van de zeespiegel is in 2014 een methodiek vastgesteld om zo objectief mogelijk de waarden ervan vast te stellen. Dit vaststellen is geen triviale zaak omdat waterstandsdata veel meer informatie bevatten dan alleen veranderingen van de middenwaterstand op zee. Jaarlijkse variatie in wind en getij zijn daar de belangrijkste van. De analyse van de data (cf. de afgesproken methodiek) leidde tot vorig jaar tot de conclusie dat een constante trend de beste beschrijving geeft, gegeven de doelen van de Zeespiegelmonitor. Vanaf dit jaar is dit niet meer zo. De zeespiegel langs de Nederlandse kust kan nu het best beschreven worden door een combinatie van twee lijnen. De eerste lijn loopt tot 1993 met een gemiddelde jaarlijkse stijging van 1,8 mm/jaar. De tweede lijn loopt vanaf 1993 met een gemiddelde jaarlijkse stijging van 2,9 mm/jaar.

Bovenstaande figuur illustreert meerdere belangrijke boodschappen van dit rapport:

-   Ook in Nederland moeten we nu rekening houden met een zeespiegelstijging in de komende decennia die hoger is dan de trend die er vorige eeuw was.
-   De nieuwe datapunten, van de laatste drie, vier jaar, maakten het doorslaggevende verschil. Ze liggen allemaal boven de historische trendlijn uit de vorige rapportage.
-   De data vertonen tot 1979 meer variatie (afwijkingen van het gemiddelde). Voor de waterstanden vanaf 1979 is het goed mogelijk om de invloed van de variatie door wind- en stormopzet weg te filteren. Dit ligt ten grond aan de lagere spreiding vanaf 1979 en verkleint de onzekerheid in de schatting van de zeespiegel en -trend.
-	De stand van de zeespiegel (hetgeen verwacht wordt gegeven de opgetreden variatie in windopzet en getij) was in 2021 hoger dan ooit geweest: 9.4 cm boven NAP.
(- ? De onzekerheid in de vastgestelde trend is kleiner dan de jaar op jaar variaties in de stand van de zeespiegel. ?)

figuur invoegen

De representatie van de zeespiegel door twee lijnen betekent niet er een lineaire stijging optreedt na een knik in het begin van de jaren negentig. Op basis van de meest plausibele denkmodellen over de wereldwijde stand van de zeespiegel is een langzaam opbouwende versnelling aannemelijker. Voor een bevestiging daarvan is nu nog te weinig data. Het doel van de Zeespiegelmonitor is een beeld te geven van de recente verandering en een vooruitblik van ca. 15 jaar. Hiervoor is een trend van 2,9 mm/jaar op dit moment de beste aanname. Voor een langere periode vooruit (of in de toekomst) kan deze trend niet gebruikt worden. In de stand en trend van de zeespiegel zijn dit jaar de gegevens van vijf van de zes Nederlandse stations gebruikt. Er is reden om aan te nemen dat de gegevens van/voor het station Delfzijl op dit moment niet betrouwbaar genoeg zijn door de snelle bodemdaling aldaar. De overige stations (Harlingen, Den Helder, IJmuiden, Hoek van Holland, Vlissingen) vertonen vergelijkbare trends in stijging sinds 1993, van tussen 2,3 en 3,3 mm per jaar. Dit is een veel coherenter beeld dan op basis van de data voor 1993. De verschillen tussen de Nederlandse stations lijken weinig tot geen implicaties te hebben voor de inschatting van de zeespiegelstijging op lange termijn. Het is aannemelijk dat hiervoor de mondiale zeespiegelstijging het dominante signaal is.

}
\deltarestitle
%------------------------------------------------------------------------------

% Text body

$body$

%------------------------------------------------------------------------------
\end{document}
